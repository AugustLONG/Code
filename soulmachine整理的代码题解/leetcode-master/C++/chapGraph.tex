\chapter{图}

无向图的节点定义如下:
\begin{Code}
// 无向图的节点
struct UndirectedGraphNode {
    int label;
    vector<UndirectedGraphNode *> neighbors;
    UndirectedGraphNode(int x) : label(x) {};
};
\end{Code}


\section{Clone Graph} %%%%%%%%%%%%%%%%%%%%%%%%%%%%%%
\label{sec:clone-graph}


\subsubsection{描述}
Clone an undirected graph. Each node in the graph contains a \code{label} and a list of its \code{neighbours}.


OJ's undirected graph serialization:
Nodes are labeled uniquely.

We use \code{\#} as a separator for each node, and \code{,} as a separator for node label and each neighbour of the node.
As an example, consider the serialized graph \code{\{0,1,2\#1,2\#2,2\}}.

The graph has a total of three nodes, and therefore contains three parts as separated by \code{\#}.
\begin{enumerate}
\item First node is labeled as 0. Connect node 0 to both nodes 1 and 2.
\item Second node is labeled as 1. Connect node 1 to node 2.
\item Third node is labeled as 2. Connect node 2 to node 2 (itself), thus forming a self-cycle.
\end{enumerate}

Visually, the graph looks like the following:
\begin{Code}
       1
      / \
     /   \
    0 --- 2
         / \
         \_/
\end{Code}


\subsubsection{分析}
广度优先遍历或深度优先遍历都可以。


\subsubsection{DFS}
\begin{Code}
// LeetCode, Clone Graph
// DFS,时间复杂度O(n),空间复杂度O(n)
class Solution {
public:
    UndirectedGraphNode *cloneGraph(const UndirectedGraphNode *node) {
        if(node == nullptr) return nullptr;
        // key is original node,value is copied node
        unordered_map<const UndirectedGraphNode *,
                            UndirectedGraphNode *> copied;
        clone(node, copied);
        return copied[node];
    }
private:
    // DFS
    static UndirectedGraphNode* clone(const UndirectedGraphNode *node,
            unordered_map<const UndirectedGraphNode *,
            UndirectedGraphNode *> &copied) {
        // a copy already exists
        if (copied.find(node) != copied.end()) return copied[node];

        UndirectedGraphNode *new_node = new UndirectedGraphNode(node->label);
        copied[node] = new_node;
        for (auto nbr : node->neighbors)
            new_node->neighbors.push_back(clone(nbr, copied));
        return new_node;
    }
};
\end{Code}


\subsubsection{BFS}
\begin{Code}
// LeetCode, Clone Graph
// BFS,时间复杂度O(n),空间复杂度O(n)
class Solution {
public:
    UndirectedGraphNode *cloneGraph(const UndirectedGraphNode *node) {
        if (node == nullptr) return nullptr;
        // key is original node,value is copied node
        unordered_map<const UndirectedGraphNode *,
                            UndirectedGraphNode *> copied;
        // each node in queue is already copied itself
        // but neighbors are not copied yet
        queue<const UndirectedGraphNode *> q;
        q.push(node);
        copied[node] = new UndirectedGraphNode(node->label);
        while (!q.empty()) {
            const UndirectedGraphNode *cur = q.front();
            q.pop();
            for (auto nbr : cur->neighbors) {
                // a copy already exists
                if (copied.find(nbr) != copied.end()) {
                    copied[cur]->neighbors.push_back(copied[nbr]);
                } else {
                    UndirectedGraphNode *new_node =
                            new UndirectedGraphNode(nbr->label);
                    copied[nbr] = new_node;
                    copied[cur]->neighbors.push_back(new_node);
                    q.push(nbr);
                }
            }
        }
        return copied[node];
    }
};
\end{Code}


\subsubsection{相关题目}
\begindot
\item 无
\myenddot

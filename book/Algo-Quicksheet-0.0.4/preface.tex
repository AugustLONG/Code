%%%%%%%%%%%%%%%%%%%%%%preface.tex%%%%%%%%%%%%%%%%%%%%%%%%%%%%%%%%%%%%%%%%%
% sample preface
%
% Use this file as a template for your own input.
%
%%%%%%%%%%%%%%%%%%%%%%%% Springer %%%%%%%%%%%%%%%%%%%%%%%%%%

\preface
\section*{Introduction}
This quicksheet contains many classical equations and diagrams for algorithm, which helps you quickly recall knowledge and ideas in algorithm.\\

This quicksheet has three significant advantages:
\begin{enumerate}
\item Non-essential knowledge points omitted
\item Compact knowledge representation
\item Quick recall
\end{enumerate}
\section*{How to Use This Quicksheet}
You should not attempt to remember the details of an algorithm. Instead, you should know:
\begin{enumerate}
\item What problems this algorithm solves.
\item The benefits of using this algorithm compared to others.
\item The important clues of this algorithm so that you can derive the details of the algorithm from them.
\end{enumerate}
Only dives into the code when you is unable to reconstruct the algorithm from the hits and the important clues. 

\vspace{\baselineskip}
\begin{flushright}\noindent
At GitHub, June 2015\hfill {\it github.com/idf} \\
\end{flushright}
